\documentclass[10pt]{article}
\usepackage{latexsym}
\usepackage{fullpage}
%\pagestyle{fancyplain}
\setlength{\textheight}{10.0in}
\begin{document}
\thispagestyle{empty}
Professor Fearing ~~~~~~~~~~ EECS192/Project Proposal v 1.1 ~~~~~~~~~~~ Spring 2015\\
{\bf Due by 5 pm Fri. Feb. 6 under door of 725 S.D. Hall.} 


The purpose of the project proposal is to give you early feedback on
your design ideas, to help to steer you away from some of the common
design flaws. A good motto to keep in mind is
``measure twice, drill once''. The pace of the rest of the course is
quite fast, and now is the time to plan what equipment you want, and
where it should be mounted. 
The project proposal is a non-trivial amount of work; we
would like to see 10 hours of effort per team member. Please
read the NatCar rules from the UC Davis site.
Here is an outline for you to follow:\\

{\bf 1. Overall Strategy (1 page) (20\%)}

1.1
Think about ``winning''
strategies. A ``winning'' strategy is one which results in
a functional, thoroughly tested, and reliable vehicle with a minimum
of effort. Separately list essential and 
``wouldn't it be great to have'' features. A good rule of thumb
is that you would like to be 80\% complete with 20\% of budget.\\

1.2
Now speed is of course an important consideration,
but speed without stability and robustness won't get you far.
Will your vehicle rely on raw speed, expert navigation,
expert braking, etc?  How does your strategy affect the types of sensors and
control strategies you will use?  \\

1.3
List anticipated sensor needs and
how they will be used. ({\bf Estimate number of lines of digital and analog
IO you will use.})\\

{\bf 2. Hardware Design }

{\bf 2.1 Attachments to Vehicle (one page) (5\%)}

List every attachment you anticipate adding to your vehicle. (You are
not committing to adding these devices, just leaving room for them if
desired). Some possible things to attach: CPU board, IO board, 
battery pack, DC-DC converter, user interface, sensors,
etc.

{\bf 2.2 Detailed Mechanical Drawings of Vehicle (2 pages) (20\%)}\\
Draw a detailed layout of the
vehicle with attachments from section 2.1. Include dimensions of
circuit boards and vehicle. Expected detail is to level of screw
holes for what you need to mount. 
Show location of flag mounting post, emergency stop switch,
and battery pack.
Do some rough sketches first: can you change the battery pack
without unscrewing anything? 
Are
switches for mode/power/reset readily accessible? Have you left room
for expansion circuitry components if needed?
(You can use ppt to sketch locations on top of provided drawings.)

Include a labelled photo which indicates good locations for mounting
an encoder wheel for speed sensing.


{\bf 2.3 List of Special Materials (0.5 page) (5\%)}

Do you need metal or plastic
brackets? Metal or plastic plates?  Special switches or connectors? If
there is enough interest, we can arrange car pools to TAP plastics or
Fry's. We also will put together periodic class mail orders to Digikey.

{\bf 2.4 Motor Drive Circuitry (1 page) (30\%)}

Show a detailed schematic for your motor drive circuitry, including values and
pin \#s. Be sure to include protection logic (if needed), e-stop, and
logic level shift (if needed).
Draw expected parts layout. Is there room for heatsinks?
(See hand out for example of a detailed schematic/parts layout.)\\

{\bf 3. Software Strategy (Confidential) (1 page) (20\%)}


3.1 Describe in some detail what you expect a good software strategy to be.
How much effort will be extended on for example, precise steering,
precise speed distance/velocity, error recovery? List those things you need
to understand more completely before you will be able to design
your software. 

3.2 State your assumptions about what quality of sensors
you are assuming. Consider geometry and field of view for line sensor(s).

3.3 Make a first pass, high-level block diagram of
the software system. (A block diagram should be like Fig. 5
in Thrun et al.~\cite{Thrun06} showing
functional modules and their interconnection, {\bf not a flow chart}.) 

\begin{small}
\bibliographystyle{IEEEtranS}
\begin{thebibliography}{10}
{
\bibitem{Thrun06} 
S. Thrun et al.,
``Stanley: The robot that won the DARPA Grand Challenge.''
{\em Journal of Field Robotics,} 23(9), 661-692, 2006.
}
\end{thebibliography}
\end{small} 
\end{document}
