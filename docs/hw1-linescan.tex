% latex 2.09 header
%\documentstyle[twoside,titlepage,fullpage,fancyheadings]{article}

% latex2e header
\documentclass[11pt]{article}
\usepackage{latexsym}
\usepackage{fullpage}
%\usepackage{fancyheadings}
\usepackage{pstricks}
% \usepackage{psfig}
%\pagestyle{fancyplain}


%\input{lab-fmt.tex}
% \input epsf.sty
\setlength{\textheight}{10.0in}
\begin{document}
\thispagestyle{empty}
Professor Fearing ~~~~~~~~~~ EECS192/Assignment \#1 v1.00~~~~~~~~~~ Spring 2015\\
{\bf Due at beginning of class Tues Mar. 17.} \\

(One assignment per team.) \\

The purpose of this assignment is to allow you to prototype 
track finding algorithms with a high level language before
implementing in C on the KL25Z processor. You are free to use
signal processing libraries, floating point, etc. One goal is
to find the best performance which can be obtained with unlimited
processing capability. However, remember that your line finding algorithm
will need to work in real-time on a 40 MIPS processor. \\


Typical strategies to use for finding the track include:
\begin{enumerate}
\item
Frame subtraction and peak detection
\item 
smoothing followed by gradient detection (e.g. difference of Gaussians
approximation to the Laplacian)
\item
curve fitting, e.g. cubic spline or $\tanh^{-1}$.
\end{enumerate}
\vspace{0.3in}

3 sets of line scan data (for a black stripe on a white background)
is provided on Piazza for EE192 under ``Resources''.
(Please note the Freescale rules for the track changed.)
A partial iPython notebook {\tt linescan-HW1.ipynb} is provided.

Complete the function {\tt find\_track(linescans)} which takes as input
$n$ frames of linescans of 128 values in the range $0 ... 65535$ and returns:\\

a) {\tt track\_center\_list} which is a length $n$ list of the index in the range $0 ... 127$ corresponding
to the center of the track in each frame.\\

b) {\tt track\_found\_list} which is a length $n$ list of booleans, {\bf True} if the track is visible for a particular frame.\\

c) {\tt Cross\_found\_list} which is a length $n$ list of booleans, {\bf True} if a crossing is present for a particular frame.\\

Upload your completed iPython notebook to bcourses.

Your python function will be tested against another data set
taken under similar conditions on a similar track.

Directions for installing iPython can be found on the EE123 web page at:\\
{\tt http://www-inst.eecs.berkeley.edu/\~{}ee123/sp15/python.html}.

\end{document}
